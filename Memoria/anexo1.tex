	\chapter*{Anexo 1 PCB} % si no queremos que añada la palabra "Capitulo"
	\addcontentsline{toc}{chapter}{Anexo 1} % si queremos que aparezca en el índice
	\markboth{Anexo 1}{Anexo 1} % encabezado
	
		Para el dise\~no de la placa PCB se us\'o la herramienta software de c\'odigo abierto KiCad, plataforma para la creaci\'on y dise\~no de esquem\'aticos de circuitos electr\'onicos y sus correspondientes archivos de fabricaci\'on de placas de circuito impreso. El hecho de ser una plataforma de c\'odigo abierto lo hace perfecto para proyectos orientados a la creaci\'on de hardware electr\'onico.
		
		Kicad no tiene limitaci\'on del tama\~no de la placa y puede manejar facilmente hasta 16 capas de cobre y hasta 12 capas t\'ecnicas. Puede generar todos los ficheros necesarios para la fabricaci\'on de placas de circuito impreso, tales como: Gerber files, component locantion files...
		
		Podemos descargar KiCad desde http://www.kicad-pcb.org, donde se nos descargar\'a un ejecutable que instalar\'a en la ruta deseada las herramientas de KiCad.
		
		Una vez ha sido instalado KiCad, lo primero que debemos realizar es dise\~nar el Esquem\'atico de nuestro proyecto. Para ello detallar\'e a continuaci\'on los pasos que segu\'i y el restultado que obtuve.
		
		Los pasos seguidos fueron los siguientes:
		
		\begin{itemize}	
			
			\item Lo primero de todo es crear un nuevo proyecto y guardarlo con el nombre que deseemos.
			
			\item Una vez ha sido creado el proyecto debemos pulsar sobre EESchema para comenzar a crear el esquem\'atico del proyecto.
			
			\item Algunos componentes se encuentran en las librer\'ias ya instalados, otros como el de la placa Arduino lo pude descargar de Internet y posteriormente asoci\'andolo a su footprint.
			
			\item Creados todos los componentes proced\'i a conectarlos con cables, incluyendo las fuentes de alimentaci\'on, las tierras y definir sin conexi\'on los pines que no utilizar\'e.
			
			\item Una vez el esquem\'atico ha sido terminado hay que proceder a enlazar todos los componente del esquem\'atico con Cvpcb.
			
		\end{itemize}
		
		El esquem\'atico desarrollado es el siguiente:
	
		\begin{figure}[h]
			\centering
			\includegraphics[9cm,0cm][10cm,10cm]{imagenes/EsquematicoPCB.jpg}
			\caption{Esquem\'atico dise\~no PCB}
			\label{contexto:figura}
		\end{figure}
		
		Ahora podemos usar la herramienta PCBnew, en la cual dise\~naremos el boceto de nuestra PCB final.
		
		Los pasos a seguir son los siguientes:
		
		\begin{itemize}	
			
			\item En primer lugar debemos fijar el tama\~no de la p\'agina a A4 y tutularla con el nombre de nuestra PCB.
			
			\item En segundo lugar debemos fijar el tamaño el clearance, ancho de las v\'ias, separaci\'on m\'inima de las v\'ias, etc.
			
			\item Accedemos al NetList, para exarminarlo y leer el Netlist actual para cargar todos los footprints asociados.
			
			\item Una vez han sido cargados, debemos colocarlos de una forma ordenada, teniendo en cuenta donde vamos a ubicar cada componente y que no haya posibles cruces de pistas ya que solo vamos a imprimir la placa a una sola capa.
			
			\item Con ''Edge Cuts'' fijaremos el tama\~no de la placa, adem\'as escogeremos la capa F.Cu con la cual creamos las v\'ias que conectar\'an todos los pines.
			
			\item El resto de la placa no usada se rellenar\'a con la capa ''F.Cu'' con la finalidad de  ser el plano de masa que ir\'a unido a los pines tierra. Debemos dejar un espacio sin rellenar, que ser\'a donde estar\'a la antena del dispositivo bluetooth, para evitar que se apantalle y se reduzca de forma considerable el alcance de dicha antena.
			
			\item Una vez ha sido creado y unido todo debemos realizar un chequeo de la PCB dise\~nada, lo que en nuestra herramienta se reconoce como ''Comenzar DRC''. Realizado dicho chequeo debemos comprobar que la ''Lista No Conectados'' se encuentra vac\'ia.
			
		\end{itemize}
		
		\
		\\
		\
		
		Realizado todo esto podemos ver el resultado de la placa que hemos diseñado con PCBnew en la siguiente imagen.
		
		\begin{figure}[h]
			\centering
			\includegraphics[8cm,0cm][9cm,14cm]{imagenes/DisenoPCB.jpg}
			\caption{Dise\~no PCB en PCBnew}
			\label{contexto:figura}
		\end{figure}
		
		Adem\'as si lo deseamos podemos ver el resultado en 3D, accediendo a la pesta\~na Ver y pinchando sobre Visualizaci\'on 3D. Obteniendo el siguiente resultado.
		
		\begin{figure}[h]
			\centering
			\includegraphics[10cm,0cm][8cm,14cm]{imagenes/3D.jpg}
			\caption{Modelo 3D PCB}
			\label{contexto:figura}
		\end{figure}
		
		Adem\'as se nos ofrece la posibilidad de generar los ficheros Gerbers. Si queremos generarlos debemos seguir los siguientes pasos.
		
		\begin{itemize}	
			
			\item Desde PCBnew accedemos a Archivo y despues Plot.
			
			\item Elegimos ''Gerber'' en el desplagable de ''Plot Format'' y elegimos el directorio donde deseamos guardar los archivos.
			
			\item Elegimos las capas que deseamos dibujar para una placa de una sola capa.
			
			\item Hacemos click sobre ''Trazar'' para generar los Gerbers y sobre ''Generate Drill File'' para generar el archivo de drills.
			
			\item Podemos acceder a ver los archivos generados pulsando sobre el icono del GerbView.
			
		\end{itemize}

	
	\newpage
	$\ $