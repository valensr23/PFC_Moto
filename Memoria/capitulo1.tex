\chapter{Introducción}\label{cap.introduccion}
	\pagenumbering{arabic} % para empezar la numeración con números
	
		En este capítulo pretendemos situar al lector en el contexto del proyecto, en que campo vamos a trabajar, motivación y los objetivos del proyecto para que se pueda entender con mayor facilidad el trabajo realizado y los objetivos a conseguir.
		
		Se busca, por tanto, que con la lectura de este capítulo pueda entenderse el resto de esta memoria.
		
	\section{Motivación}
		
		Desde la invención de la motocicleta por Sylvester Howard Roper en 1867, el mundo del motor ha evolucionado desde el motor de cilindros a vapor hasta los motores actuales. Diferentes tipos de motocicletas con distintas cilindradas inundan nuestras calles a día de hoy.
		
		Como podemos ver al salir a la calle las motocicletas no son el único vehículo que nos encontramos, también podemos encontrar coches, camiones y sus variantes en forma, motor, tama\~no y número de ruedas. Lo que si podemos observar es que la moticleta es el vehículo menos seguro frente a los mencionados, no solo por dimensiones sino porque en caso de accidente es el cuerpo del motorista quien sufre el impacto y no la carrocería como en los otros vehículos.
		
		Con el paso de los a\~nos la seguridad en vehículos con motor ha aumentado progresivamente. Este crecimiento ha sido mayor con la introducción de nuevas tecnologías, tales como sensores, dotan al vehículo de cierta inteligencia.
		
		A la hora de usar un vehículo de dos ruedas lo mas importante es nuestra seguridad, prueba de ello son los sistemas que se están implantando tanto en el propio vehículo como en cascos o chaquetas. En el caso de sufrir un accidente el tiempo juega en nuestra contra, una rápida actuación de los servicios sanitarios puede salvarnos la vida en caso de que lo necesitemos.
		
		Al a\~no se producen mas de 2000 accidentes de moto, el 72\% de esos accidentes precisa atención médica. Algunos vehículos de alta gama ya poseen un sistema de seguridad en caso de sufrir en accidente, sistema que será obligatorio en Espa\~na a partir del 31 de Marzo de 2018 en coches. Quedando disponible realizar un sistema parecido para motos, el cual podría llegar a ser obligatorio en un futuro.
		
		Hoy en día es difícil encontrar una persona que no posea un smartphone, una herramienta que no solo sirve para llamar, leer la prensa o encontrar una ubicación, sino que con la posibilidad del desarrollo de aplicaciones podemos convertir nuestro smartphone en una herramienta que nos ayude en cualquier tarea, circunstancia o incluso que nos salve la vida.
		
		Si combinamos la tecnología de nuestro smartphone con un sistema electrónico que pueda ser integrado en una moto, ya sea desde el momento de su fabricación o un extra, aumentariamos la seguridad de todo motorista.
		
		Es cierto que no disminuiriamos el números de accidentes que puedan ocurrir en carretera, pero si disminuiríamos el tiempo de reacción de los servicios sanitarios, aumentando las probabilidades de atender a una persona en el menor tiempo posible.
		
	\section{Descripción del proyecto}
	
		Como se ha indicado, es muy importante el tiempo de reacción de los servicios de emergencias, facilitando ademas nuestra ubicación ayudamos a que nos encuentren con facilidad en caso de estar en algún lugar de difícil acceso, que no hayamos podidos llamar por nuestros medios o que no hubiera testigos de dicho accidente que pudieran llamar a emergencias para informar de la gravedad de lo ocurrido.
		
		Además debemos tener en cuenta que este sistema funciona solo, no necesita de la intervención del motorista una vez el sistema ha sido iniciado en caso de accidente.
		
		Para que este sistema sea efectivo es preciso que el motorista circule con un smartphone encendido, conectado vía bluetooth al sistema electrónico integrado en la motocicleta y con la aplicación iniciada.
		
		Este proyecto pretende ser un sistema completo y autónomo que se componga de un sistema electrónico, compuesto por un giróscopo, magnetómetro y acelerómetro, capaz de transmitir vía bluetooth a un smartphone todos los parámetros necesarios para que la aplicación desarrollada ejecute el algoritmo implementado.
				
		Por ello podemos decir que los elementos clave para aumentar la seguridad en un vehículo de dos ruedas y reducir el tiempo de intervención de los servios de emergencias en este proyecto serían:

		\begin{itemize}
			\item Sensores.
			\item Bluetooth.
			\item Smartphone.
		\end{itemize}
		
		Además, es necesario que el smartphone posea GPS, para una vez los datos han sido recibidos y almacenados, podamos operar con ellos en nuestro algoritmo, el cual nos comprobará si hemos sufrido un accidente en base a nuestro ángulo de inclinación y la velocidad a la que circulamos. En caso que el algoritmo detecte haber sufrido un accidente enviará un SMS a los servicios de emergencias indicándoles la ubicación recogida por el GPS.
	
	
	\section{Objetivos}
	
		Para llevar a cabo este proyecto, debemos tener como objetivos los siguientes:
		
		\begin{itemize}
			\item Estudio de las bibliografía acerca de la calibracón de los sensores y de los algoritmos de cálculo de Pitch, Roll y Yaw.
			
			\item Diseño un sistema de sensado inercial microcontrolado basado en arduino.
			
			\item Montaje de la mota sensora y dispositivo bluetooth HC-05. Montar la mota sensora y el módulo HC-05, y tenerlos listos para su uso, realización de programas básicos de prueba y verificación de que funcionan correctamente.
			
			\item Diseño de una aplicación móvil para la comunicación con los sensores y su comunicación exterior.
			
			\item Testeo del sistema en diferentes condiciones. Realización de las pertinentes pruebas para comprobar el correcto funcionamiento del sistema desarrollado. Análisis a diferentes velocidades, diferentes ángulos de inclicación del sistema electrónico que irá integrado en la moto. Además de pruebas del sistema en diferentes circunstancias meteorológicas para comprobar el alcande del dispositivo bluetooth.
			
			\item Desarrollo de un prototipo final en placa de circuito impreso.
			
		\end{itemize}
		
		En este proyecto, se pretende entrar en contacto con el dise\~no y programación de una placa Arduino UNO, la interpretación de los datos que proporciona una unidad inercial, que está compuesta por dos sensores, por un lado, el acelerómetro y magnetómetro LSM303DLHC y por otro el giróscopo L3GD20, establecer la comunicación Bluetooth entre la placa Arduino UNO y un dispositivo Android y la programación y dise\~no de la aplicación móvil en Android.
		
		Para ello conectaremos el Pololu MinIMU-9 v2 y el dispositivo Bluetooth HC-05 a la placa Arduino UNO, programaremos nuestra placa Arduino para que reciba los datos del sensor y los envíe mediante Bluetooth al dispositivo Android.
		
		Posteriormente debemos programar una aplicación Android en nuestro Smartphone que sea capaz de conectarse al dispositivo HC-05 y leer la información recibida, esta aplicación será la encargada de ejecutar el algoritmo y comprobaciones pertinentes. En caso de haber sufrido un accidente se le notificara al número de emergencias 112 de nuestra ubicación via SMS.
		
				
	\newpage
	$\ $