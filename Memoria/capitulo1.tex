\chapter{Introducci\'on}\label{cap.introduccion}
	\pagenumbering{arabic} % para empezar la numeración con números
		En este cap\'itulo pretendemos situar al lector en el contexto del proyecto, en que campo vamos a trabajar, motivaci\'on y los objetivos del proyecto para que se pueda entender con mayor facilidad el trabajo realizado y los objetivos a conseguir.
		
	\section{Motivaci\'on}
		Desde la invenci\'on de la motocicleta por Sylvester Howard Roper en 1867, el mundo del motor ha evolucionado desde el motor de cilindros a vapor hasta los motores actuales. Diferentes tipos de motocicletas con distintas cilindradas inundan nuestras calles a d\'ia de hoy.
		
		Como podemos ver al salir a la calle las motocicletas no son el \'unico veh\'iculo que nos encontramos, tambi\'en podemos encontrar coches, camiones y sus variantes en forma, motor, tama\~no y n\'umero de rudas. Lo que si podemos observar es que la moticleta es el veh\'iculo menos seguro frente a los mencionados, no solo por dimensiones sino porque en caso de accidente es el cuerpo del motorista quien sufre el impacto y no la carrocer\'ia como en los otros veh\'iculos.
		
		Con el paso de los a\~nos la seguridad en veh\'iculos con motor ha aumentado progresivamente. Este crecimiento ha sido mayor con la introducci\'on de nuevas tecnolog\'ias, tales como sensores, dotan al veh\'iculo de cierta inteligencia.
		
		Hoy en d\'ia es dif\'icil encontrar una persona que no posea un smartphone, una herramienta que no solo sirve para llamar, leer la prensa o encontrar una ubicaci\'on, sino que con la posibilidad del desarrollo de apps podemos convertir nuestro smartphone en una herramienta que nos ayude en cualquier tarea, circunstancia o incluso que nos salve la vida.
		
		La combinaci\'on de un veh\'iculo con sensores y comunicacion Bluetooth con un smartphone es el punto clave para que este proyecto pueda ser llevado a cabo.
		
		Por ello podemos decir que los elementos clave para aumentar la seguridad en un veh\'iculo de dos ruedas en este proyecto ser\'ian:

		\begin{itemize}
			\item Sensores.
			\item Bluetooth.
			\item Smartphone.
		\end{itemize}
	
	\section{Sensores}
		En este proyecto se usar\'a una mota sensora que se compone de tres sensores:
		\begin{itemize}
			\item Aceler\'ometro: es el encargado de medir aceleraciones, mide la aceleraci\'on asociada con el fen\'omeno de peso experimentado por una masa de prueba que se encuentra en el marco de referencia del dispositivo.
			
			\item Gir\'oscopo: es el encargado de medir, mantener o cambiar la orientaci\'on en el espacio de algún aparato o veh\'iculo. Est\'a formado por un cuerpo con simetr\'ia de rotaci\'on que gira alrededor del eje de dicha simetr\'ia.
			
			\item Magnet\'ometro: es el encargado de medir en fuerza o direcci\'on la señal magn\'etica de una muestra. 
		\end{itemize}
	
	\section{Objetivos}
		En este proyecto, se pretende entrar en contacto con el dise\~no y programaci\'on de una placa Arduino UNO, la interpretaci\'on de los datos que proporciona el Pololu MinIMU-9 v2, establecer la comunicaci\'on Bluetooth entre la placa Arduino UNO y un dispositivo Android y la programaci\'on y dise\~no de la app m\'ovil en Android.
	
		Para ello conectaremos el Pololu MinIMU-9 v2 y el dispositivo Bluetooth HC-05 a la placa Arduino UNO, programaremos nuestra placa Arduino para que reciba los datos del sensor y los env\'ie mediante Bluetooth al dispositivo Android.
		
		Posteriormente debemos programar una app Android en nuestro Smartphone que sea capaz de conectarse al dispositivo HC-05 y leer la informaci\'on recibida, esta app ser\'a la encargada de ejecutar el algoritmo y comprobaciones pertinentes. En caso de haber sufrido un accidente se le notificara al n\'umero de emergencias 112 de nuestra ubicaci\'on via SMS.
		
		Para llevar a cabo este proyecto, el trabajo a realizar se divide en los siguientes objetivos:
		
		\begin{itemize}
			\item Instalaci\'on de entorno Arduino y familiarizarse con dicha programaci\'on.
			
			\item Estudio de las bibliograf\'ia acerca de la calibrac\'on de los sensores y de los algoritmos de c\'alculo de Pitch, Roll y Yaw.
			
			\item Montaje de la mota sensora y HC-05.
			
			\item Estudio de programaci\'on Android en Eclipse. Con la correspondiente app que ejecute nuestro algoritmo.
			
			\item Pruebas del sistema en diferentes condiciones.
			
			\item Desarrollo de PCB's.
		\end{itemize}
		
		
	\newpage
	$\ $