	\chapter{Conclusiones}\label{cap.conclusiones}
	
	En este capítulo se lleva a cabo una reflexión sobre el trabajo realizado, se proponen posibles mejoras que podrían realizarse en un futuro en función de los avances tecnológicos y de cara a las necesidades que surjan si procedemos a comercializar este producto.
	
	\section{Conclusiones}
	
		En el presente proyecto se ofrece una monitorización a distancia del sistema electrónico que vamos a instalar en la moto. El sistema de compone de dos dispositivos, uno de ellos será el sistema electrónico que se implantará en la moto compuesto principalmente por la placa Arduino Uno y un smartphone con la aplicación MotoSafe que poseerá la persona encargada de conducir la motocicleta. A través del bluetooth ambos sistemas podrán interactuar, de forma que el smartphone sea el encargado de recibir la información transmitida por la mota sensora del sistema electrónico implantado en la moto.
		
		Este sistema ha sido dise\~nado en función a unos requisitos de movilidad, independencia, tiempo real y tolerancia a errores, entre otros, el dise\~no de una PCB eficiente y útil, la cual nos ayudará a evitar el uso de cables en nuestro sistema electrónico. Además el smartphone será el encargado se monitorizar el sensor a través de la información recibida y aplicar el algoritmo para detectar si ha habido un accidente.
		
		Cualquier persona con un smartphone podrá poseer la aplicación, solo necesitará la implantación del sistema electrónico en su moto y podrá usar este sistema de seguridad en cualquier momento. Diariamente todas las personas llevan un smartphone encima, lo cual hace que este sistema de seguridad sea cómodo de llevar al estar ya acostumbrados.
		
		Además hay que destarcar la utilización tanto de hardware como de software libre, Android y Arduino, que proyectan un sistema extensible por terceras personas, facilitando así las líenas de trabajo futuras y permitiendo la potencial mejora mediante la colaboración de todos.
	
	\section{Líneas de trabajo futuras}
	
		El dise\~no de este proyecto deja abierta la puerta a múltiples líneas de desarrollo futuras, ya sean de forma interna o a través de terceros. Algunas de estas líneas se detallan a continuación:
		
		\begin{itemize}	
			\item Aunque el sistema electrónico a implantar en la moto ocupa un espacio reducido, una línea de mejora importante será la posibilidad de reducir el tama\~no de dicho sistema. Para ello se seleccionarián los componentes que usamos de la placa Arduino y desarrollar pos nosotros mismos una placa con dichos componentes, con la posibilidad de reducir gastos y hacer un sistema mas robusto y estable. 
			
			\item Una línea de mejora en el desarrollo software sería la mejora del algoritmo de detección de accidente, posibilitando la variación del ángulo de inclinación máximo en función de la cilindrada, peso y dimensiones de la la propia moto.
			
			\item Por último, se plantea la idea de mejorar el protocolo de comunicación mediante la integración de un sistema de recuperación o corrección de errores. El más simple a implementar sería el bit de paridad, pero existen otros mucho mas eficientes, aunque más complejos, para la implementación de un protocolo adecuado para el manejo de información del sistema.
			
		\end{itemize}
		
		Además los objetivos que debemos cumplir de cara a la comercialización de este proyecto serían los siguientes:
		
		\begin{itemize}
			\item Acabado sistema electrónico: debe ser un sistema de tamaño reducido, de facil ingración a una moto y con una entrada de alimentación desde la batería.
			
			\item Comodidad de uso de cara al usuario: el individuo que disponga de dicho proyecto, solo debe preocuparse de emparejar el bluetooth de su smartphone con el bluetooth del sistema electrónico, despues de ello, solo debe encender la aplicación antes de circular con su moto y apagarla una vez haya llegado a su destino. 
			
			\item Monitorización en tiempo real: se requiere de una comunicación en tiempo real entre smartphone y sistema electrónico, para que se pueda actuar en la mayor brevedad posible.
			
			\item Movilidad: la aplicación puede ser instalada en cualquier dispositivo portatil Android que posea bluetooth, GPS y pueda enviar SMS, por lo que debe poseer una tarjeta SIM que posibilite el uso de la red telefónica.
			
			\item Disponibilidad, robustez y eficiencia: el sistema debe ser tolerante y recuperable a fallos. Por un ejemplo un fallo puede ser la pérdida de comunicación bluetooth momentanea, por lo que se reconoce como un fallo puntual y no como un accidente. La transmisión de datos debe diseñarse con una estructura lo mas eficiente posible, con una buena sincronización. Además se debe desarrollar una interfaz de la aplicación intuitiva.
			
			\item Interpretabilidad de los datos: no solo la monitorización en tiempo real es necesaria para el correcto funcionamiento del algoritmo implementado. Una buena y rápida interpretabilidad de los datos es fundamental. En esta versión de prueba los datos se mostrarán en diferentes TextView, posibilitando así la interpretación de los datos y poder ver el comportamiento del algoritmo en función de los datos recibidos.
			
		\end{itemize}


	\newpage
	$\ $