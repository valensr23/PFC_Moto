	\chapter{Dise\~no}\label{cap.desarrollo}
	
	En este cap\'itulo se va a explicar de forma m\'as detallada las caracter\'isticas y el funcionamiento de todos los elementos de los que se har\'a uso para llevar a cabo el desarrollo de este proyecto. Adem\'as se detallar\'a el desarrollo hardware, firmware y software, explicando con detalle la configuraci\'on electr\'onica y se presentar\'a la soluci\'on software implementada.
	
	\section{Hardware}
	
		Los dispositivos hardware que vamos a usar y sus configuraciones son los siguientes.
		
		\subsection{Arduino Uno R3}
		
			Arduino Uno R3 \cite{ArduinoUno}, placa electr\'onica basada en el microcontrolador ATmega 328. Cuenta con 14 pines digitales de entrada/salida, 6 entradas anal\'ogicas, un resonador cer\'amico de 16 MHz, una conexi\'on USB, un conector de alimentaci\'on, una cabecera ICSP y un bot\'on de reinicio. Basta con conectarlo a un ordenador con un cable USB o a una bateria para empezar.
			
			\begin{figure}[h]
				\centering
				\includegraphics[5cm,0cm][11cm,11cm]{imagenes/ArduinoUno1.jpg}
				\caption{ATmega328 en Arduino Uno R3}
				\label{contexto:figura}
			\end{figure}
			
			Arduino Uno puede ser alimentado con 6 hasta 20 voltios. Si se alimenta con menos de 7 voltios, el pin encargado de suministrar 5 voltios es muy probable que suministre menos. En caso de suministrar mas de 12 voltios, el regulador de voltaje se puede sobrecalentar y da\~nar la placa. El rango de alimentaci\'on recomendado es de 7 a 12 voltios.
			
			Para programar Arduino Uno se puede usar el software de Arduino. Software desde el que podremos cargar nuestros programas en la placa Arduino a trav\'es de un cable USB, Arduino Uno cuenta con una memoria de 2KB de SRAM y 1 KB de EEPROM.
			
			ATmega328 \cite{ATmega328} es un microcontrolador creado por Atmel y que pertenece a la serie megaAVR. ATmega328 es un circuito integrado de alto rendimiento que est\'a basado en un microcontrolador RISC, combinando 32 KB ISP flash de memoria con la capacidad de leer y escribir, 1 KB de memoria EEPROM, 2 KB de SRAM, 23 l\'ineas de E/S de prop\'osito general, 32 registros de proceso general, tres temporizadores contadores con modo de comparaci\'on, interrupciones internas y externas, programador de modo USART, una interfaz serial orientada a byte de 2 cables, SPI puerto serial, 6 canales 10 bit conversor A/D y cinco modos de ahorro de energ\'ia seleccionables por software. Opera entre 1,8 y 5,5 voltios. Este dispositivo alcanza una respuesta de 1 MIPS, balanceando consumo de energ\'ia y velocidad de proceso.
		
		\subsection{Pololu minimu 9 v.2}
		
			Pololu minimu-9 v2 \cite{Pololu}, la cual es una unidad de medici\'on inercial (IMU), que viene equipada con un L3GD20 giroscopio de 3 ejes y un LSM303DLHC con 3 ejes para el aceler\'ometro y 3 ejes para el magnet\'ometro. El m\'odulo incluye un regulador de voltaje y un circuito de desplazamiento que permite el funcionamiento de 2,5 a 5,5 V.
			
			Cuyo Schematic se puede ver la figura 3.2.
			
			\begin{figure}[h]
				\centering
				\includegraphics[6cm,0cm][11cm,17cm]{imagenes/pololuSchematic.png}
				\caption{Schematic Pololu Minimu 9 V.2}
				\label{contexto:figura}
			\end{figure}
			
		\subsection{Bluetooth HC-05}
		
			Este m\'odulo Bluetooth \cite{HC05} ofrece una mejor relaci\'on precio frente a prestaciones, es un m\'odulo maestro esclavo, no solo recibe conexiones sino que tambi\'en las genera hacia otros dispositivos Bluetooth. Posee la versi\'on V2.0+EDR, trabajando a una frecuencia de 2,4 GHz en la banda ISM, con una modulaci\'on GFSK.
			
			\begin{itemize}
				\item Soporta comando AT para ser configurado
				\item Velocidad: hasta 2,1 Mbps y S\'incrono 1Mbps/1Mbps
				\item Alcance: 10 metros 
			\end{itemize}
			
			El alcance te\'orico es de 10 metros, pero el alcance pr\'actico es de 8,5 metros aproximadamente, medidos en un espacio abierto sin inclemencias meteorol\'ogicas.
			
			El Schematic del dispositivo HC-05 podemos verlo en la figura 3.3.
			
			\begin{figure}[h]
				\centering
				\includegraphics[3cm,0cm][11cm,17cm]{imagenes/HC05Schematic.jpg}
				\caption{Schematic Bluetooth HC-05}
				\label{contexto:figura}
			\end{figure}
			
		
	\section{Firmware}
	
		En los que respecta al firmware vamos a detallar como hemos configurado cada uno de los dispositivos hardware y conectado entre s\'i. Mostrando al final un Schematic con todos los componentes conectados y su implementaci\'on sobre una protoboax.
		
		\subsection{Configuraci\'on Arduino Uno y Pololu Minimu 9 V.2}
		
			El uso de este sensor en la placa Arduino requer\'ia a\~nadir dos librer\'ias en el software que posteriormente cargar\'iamos en la placa arduino.
			
			Las dos librer\'ias usadas son:
			
			\begin{table}[H]
				\centering
				\begin{tabular}{p{4cm} p{4cm}}
					\hline
					\#include & L3G.h \\
					\hline
					\#include & LSM303.h \\
					\hline
				\end{tabular}
				\caption{Liberias cargadas en Arduino para usar la motasensora}
				\label{tabla:LibreriasMota}
			\end{table}
			
			Adem\'as hemos usado la librer\'ia \#include SoftwareSerial.h, que permite la comunicaci\'on entre los pines de la motasensora y de la placa Arduino.
			
			Una vez cargadas dichas librer\'ias inicializamos todas las variables a usar en el c\'odigo adem\'as de los pertinentes m\'etodos para tomar los valores medidos de cada sensor en los 3 ejes. En lo que respecta al c\'odigo Arduino siempre tendremos dos m\'etodos a usar, como son setup() el encargado de configurar la placa e inicializar los m\'etodos y el m\'etodo loop(), el cual ejecutar\'a las acciones que le ordenemos de forma reiterativa.
			
			En lo que respecta a las conexiones, conectamos la patilla SDA y SCL con la patilla A4 y A5 respecticamente de la placa Arduino, adem\'as de VCC y GND entre la motasensora y la placa.
			
			\begin{figure}[h]
				\centering
				\includegraphics[6cm,0cm][7cm,10cm]{imagenes/ConfMota.jpg}
				\caption{Conexiones Motasonra y Arduino}
				\label{contexto:figura}
			\end{figure}
			
			
			En lo que respecta a la configuraci\'on de la mota sensora, debemos fijarle una posici\'on inicial. Que ser\'a con respecto a la cual calculemos los \'angulos de inclinaci\'on con respecto a la vertical de la moto. Al encender este sistema electr\'onico se produce un calibrado de los angulos iniciales. 
			
		
		\subsection{Configuraci\'on Arduino Uno y Bluetooth HC-05}
	
			El uso del dispositibo bluetooth HC-05 en la placa Arduino requer\'ia a\~nadir dos librer\'ias en el software que posteriormente cargar\'iamos en la placa arduino.
			
			Las librer\'ias usadas son:
			
			\begin{table}[H]
				\centering
				\begin{tabular}{p{4cm} p{4cm}}
					\hline
					\#include & SoftwareSerial.h \\
					\hline 
					\#include & wire.h \\
					\hline 
				\end{tabular}
				\caption{Liberias cargadas en Arduino para usar Bluetooth HC-05}
				\label{tabla:LibreriasMota}
			\end{table}
			
			En nuestro c\'odigo Arduino debemos dejar indicado que el pin RXD es el 10 y el pin TDX es el 11 con el siguiente comando:
			
			SoftwareSerial BT = SoftwareSerial(10, 11); //10 RX, 11 TX.
			
			Adem\'as debemos configurar el Arduino para que el pin 10 sea de entrada y el pin 11 de salida, indicando en el PinMode si es InPut o OutPut respectivamente. Ambos componentes se deben comunicar a 9600 baudios, velocidad por defecto de funcionamiento del bluetooth HC-05, una configuraci\'on a una velocidad diferente provoca que no haya comunicaci\'on entre el dispositivo bluetooth y la placa Arduino.
			
			En lo que respecta a las conexiones, la patilla RXD y TXD del m\'odulo bluetooth se conectar\'an a las patillas 11 y 12 respectivamente de la placa Arduino, adem\'as de VCC y GND entre el m\'odulo bluetooth y la placa.
			
			En los que respecta al m\'etodo loop() de nuestro c\'odigo Arduino lo que haremos ser\'a comprobar que el m\'odulo bluetooth HC-05 se encuentra conectado a un dispositivo m\'ovil, si la respuesta es negativa permaneceremos a la espera de conexi\'on, en caso de ser afirmativa la respuesta enviaremos por dicho canal de comunicaci\'on los datos medidos en la motasensora.
			
			Dichos datos ser\'an enviados cada 500 milisegundos, periodo de env\'io que he establecido para no saturar al receptor.
			
		\subsection{Schematic conexi\'on m\'odulos y placa Arduino}
		
			En el Schematic que se mostrar\'a a continuaci\'on podemos ver de forma f\'acil y r\'apida todas las conexiones que se han implementado.
			
			\begin{figure}[h]
				\centering
				\includegraphics[6cm,0cm][7cm,9cm]{imagenes/Schematic.jpg}
				\caption{Schematic m\'odulos y Arduino conectados}
				\label{contexto:figura}
			\end{figure}
			
			El m\'odulo del centro representa a la placa Arduino, mientras que el m\'odulo a la izquiera de la imagen representa la morasensora y el m\'odulo de la derecha al dispositivo bluetooth HC-05.
			
			Una vez esto ha sido implemetado y conectado en la placa Arduino con la ayuda de una protoboax, obtenemos el siguiente resultado mostrado en la figura 3.6.
			
			\begin{figure}[h]
				\centering
				\includegraphics[6cm,0cm][7cm,11cm]{imagenes/implementacion.jpg}
				\caption{Resultado conexi\'on todos los componentes}
				\label{contexto:figura}
			\end{figure}
		
	\section{Software}
	
		En esta secci\'on proceder\'e a explicar en detalle el desarrollo de la app Android, as\'i como el diagrama de estados del algoritmo implementado y la interfaz eventual para comprobar el funcionamiento correcto de la app.
		
		\subsection{Android}
		
			Para el desarrollo de \'esta app hemos descargado el programa Eclipse ADT de Android Developer. Programa que he usado para el desarrollo \'integro de la app MotoSafe. Se debe prestar atenci\'on a partir de que versi\'on de Android deseamos implementar la app, escog\'i la versi\'on Android 4.4 ya que es la que actualmente posee mi smartphone, para una posterior comercializaci\'on debo hacer \'esta app disponible a partir de la version Android 2.3 ya que la mayor\'ia de los smartphones poseen \'esta versi\'on.
			
			En su implementaci\'on se ha usado una \'unica clase, que es la clase main, por defecto invocada al ejecutarse el programa y que contiene el bucle que realiza todas las acciones necesarias para que el programa funcione correctamente,  tal y como se muestra en la figura 3.7. 
			
			Aqui debo destacar que en el fichero main no solo contengo la clase main, sino tambi\'en la clase RecibirComando, que es la encargada de enviar por el Handler toda la informaci\'on recibida del bluetooth.
		
			\begin{figure}[h]
				\centering
				\includegraphics[5cm,0cm][4cm,10cm]{imagenes/main.JPG}
				\caption{Estructura app Android}
				\label{contexto:figura}
			\end{figure}
			
			En resumen, la clase main ser\'a la encargada de gestionar todos los datos recibidos por el Handler, comprobar el estado del GPS, Bluetooth y si ambos dispositivos se encuentran conectados o se pierde la conexi\'on. Adem\'as ser\'a la clase encargada de ejecutar el algoritmo y en caso necesario registrar la ubicaci\'on para posteriormente enviarsela v\'ia SMS a emergencias.
			
			En los pr\'oximos apartados se explicar\'a con mas detalle el funcionamiento de la app y se mostrar\'a el diagrama de estados.
		
		\subsection{Diagrama estados}
		
			En este subapartado se va a explicar el funcionamiento de la app adem\'as de un diagrama de estados representativo.
			
			A la hora de empezar a usar la app MotoSafe debemos entrar en la configuraci\'on del bluetooth de nuestro smartphone y emparejarlo con el dispositivo HC-05, la clave por defecto para su emparejamiento es ''1234''. Para poder realizar dicha emparejamiento debemos tener en cuenta que el m\'odulo bluetooth se debe encontrar encendido, para ello la motocicleta se debe encontrar arrancada o con el contacto encendido. Una vez hemos realizado dicho proceso ya podemos abrir esta app y comenzar a usarla.
			
			Al abrir la app MotoSafe encontraremos la interfaz que mostramos en la figura 3.9. Si pulsamos sobre el bot\'on OFF la aplicaci\'on se cerrar\'a y no continuar\'a ejecut\'andose en segundo plano, es decir, mata este proceso. Si por el contrario pulsamos el bot\'on ON, se iniciar\'an una serie de procesos que har\'an funcionar correctamente la app.
			
			Lo primero que har\'a ser\'a comprobar el estado del bluetooth de nuestro smartphone, en el caso de tenerlo desactivado un alert nos informa de esto y nos ofrece de posibilidad de activarlo, en caso de negarnos, la app se cerrar\'a autom\'aticamente. El siguiente paso es comprobar el estado del GPS, en el caso de tenerlo desactivado un alert nos informa de esto y nos ofrece de posibilidad de activarlo, en caso de negarnos, la app se cerrar\'a autom\'aticamente.
			
			Posteriormente y de forma autom\'atica se crear\'a un Handler que ser\'a el encargado de conectar la app con el bluetooth y poder interpretar todos los datos que recibe. En el caso de no estar emparejado el dispositivo no recibiremos datos y por tanto la app no ejecutar\'a el algoritmo implementado.
		
			\begin{figure}[h]
				\centering
				\includegraphics[8cm,0cm][10cm,15cm]{imagenes/DiagramaEstados.png}
				\caption{Diagrama de Estados}
				\label{contexto:figura}
			\end{figure}
			
			Una vez todo est\'a activado, la app est\'a lista para funcionar, para ello lo primero que har\'a ser\'a leer los datos recibidos por el Handler e interpretarlos. Nuestro algoritmo recibir\'a los datos ya interpretados, con ellos proceder\'a a calcular el \'angulo de inclinaci\'on de la motocicleta. Si este \'angulo es inferior a 55 grados volvemos a hacer los c\'alculos con los siguientes datos recibidos. Este proceso se realiza de forma peri\'odica cada 550 milisegundos. En el caso de que \'este \'angulo sea superior a 55 grados procedemos a comprobar la velocidad de la motocicleta.
			
			Para comprobar la velocidad de la motocicleta se usar\'a el GPS del smartphone, por lo que en este momento se proceder\'a al uso del GPS, para ahorrar bater\'ia no lo tendremos encendido durante todo el transcurso del desplazamiento. Si la velocidad medida es mayor a 3 metros por segundo, volvemos al principio del algoritmo y volvemos a calcular el \'angulo de inclinaci\'on de la motocicleta, en caso de que la velicidad sea menor a 3 metros por segundo procedemos a tomar a la ubicaci\'on seg\'un las coordenadas de latitud, longitud y precisi\'on.
			
			Estos datos obtenidos por el GPS ser\'an los que enviemos via SMS al n\'umero de emergencias, indic\'andoles la posici\'on actual en latitud, longitud y la precisi\'on con que mide el GPS del smartphone.
			
			En el caso de perder la se\~nal y no recibir datos, el sistema entrar\'a en un proceso de espera temporal, si no se recupera la recepci\'on de datos transcurriendo este tiempo se proceder\'a a calcular la velocidad y seguir con el disgrama de estados seg\'un la velocidad. Si durante el transcurso de ese tiempo de espera se pulta el bot\'on OFF la aplicaci\'on se cerrar\'a completamente.
				
		
		\subsection{Interfaz App}
		
			La interfaz eventual de nuestra app es la mostrada en la figura 3.9. En la cual se pueden apreciar dos botones, uno de encendido ON y otro de apagado OFF. Adem\'as disponemos de 4 TextView, los cuales usaremos para comprobar el estado actual de la app.
			
			El primer TextView muestra el \'angulo theta de inclinaci\'on, siendo su valor entre -90 y 90 grados con respecto a la vertical, dependiendo si estamos inclinados hacia la derecha o hacia la izquierda. El segundo TextView mostrar\'a dos mensajes, uno de ellos ''no me he caido'' y el otro ''es posible que me haya caido'', si el mensaje mostrado es el segundo el tercer TextView mostrar\'a ''comprobando velocidad''. Bas\'andose en el algoritmo implementado, si se ha sufrido un accidente el cuarto y \'ultimo TextView mostrar\'a el siguiente mensaje ''He sufrido un accidente y mi ubicaci\'on es: latitud 'x' longitud 'y' precisi\'on 'z' '' \'este ser\'a el mensaje que se enviar\'a v\'ia SMS a emergencias, mostrando las coordenadas calculadas con el GPS. En caso de no haber sufrido accidente no se mostrar\'a nada en este \'ultimo TextView. 
			
			\begin{figure}[h]
				\centering
				\includegraphics[5cm,0cm][7cm,17cm]{imagenes/interfaz.png}
				\caption{Interfaz app Android}
				\label{contexto:figura}
			\end{figure}

	\newpage
	$\ $
	
	\newpage
	$\ $